\chapter{Functional requirements}

All terms used in this description are defined in the Data Dictionary contained in Table \ref{tab:func_data_dictionary}.

\begin{table}[t]
\centering
\resizebox{\textwidth}{!}{%
\begin{tabular}{c|l|l|c|c|c|c}
\hline
\textbf{ID} & \multicolumn{1}{c|}{\textbf{Term}} & \multicolumn{1}{c|}{\textbf{Text Description}}                                                                                   & \textbf{Direction} & \textbf{DataType} & \textbf{Port} & \textbf{Pin} \\ \hline
I1          & B\_RESET                           & Button used to reset the system                                                                                                  & IN                 & boolean           &               &              \\ \hline
I2          & B\_ZOOM                            & \begin{tabular}[c]{@{}l@{}}Button used to change current zoom level\\ or to start the system of in Calibration Mode\end{tabular} & IN                 & boolean           &               &              \\ \hline
O1          & TRIG\_RX                           & Trigger signal for right sensor                                                                                                  & OUT                & boolean           & GPIOA         & 7            \\ \hline
O2          & TRIG\_LX                           & Trigger signal for left sensor                                                                                                   & OUT                & boolean           & GPIOB         & 15           \\ \hline
I3          & ECHO\_RX                           & Echo response of right sensor                                                                                                    & IN                 & boolean           & GPIOA         & 5            \\ \hline
I4          & ECHO\_LX                           & Echo response of left sensor                                                                                                     & IN                 & boolean           & GPIOB         & 13           \\ \hline
O3          & PWM                                & PWM used to control motor position                                                                                               & OUT                & boolean           & GPIOA         & 1            \\ \hline
O4          & LCD\_BUS                           & Bus connecting the board to LCD screen                                                                                           & OUT                & boolean           & CON3           &             
\end{tabular}%
}
\caption{Functional Requirements Data Dictionary}
\label{tab:func_data_dictionary}
\end{table}

Following we have functional requirements:


\begin{enumerate}
    \item Two {\em sensors} shall be used to detect obstacles within given ranges.
    
    \begin{enumerate}
        \item Sensors shall be positioned on a mechanical {\em arm}, mounted on top of a {\em motor}, which will change arm's angular position.
        
        \item Sensors will change {\em orientation} with the mechanical arm, along the angle range in one direction or in the opposite, scanning angles in that range in fixed steps.
        
        \item More precisely, every \SI{70}{\milli\second} the arm angular position shall change by a {\em fixed angle}, depending on the system's {\em angular speed} and current direction.
        
        \item For each position reached by the sensors, the system shall try to detect whether obstacle is present or not and if so, what is its distance.
        
        \item When the system reaches one end of the angle range moving in a certain {\em direction}, it shall start moving in the opposite direction, until the other end is reached.
        
        \item Distances of obstacles detected by each sensor shall be used to compute a triangulation of currently detected object.
        
        
    \end{enumerate}
    
    \item For each detected obstacle, the system shall show a {\em marker} on the screen.
    
    \begin{enumerate}
    
        \item Markers will be positioned inside a {\em grid}, which will represent the space around the system. The grid will allow the user to perceive the obstacle's distance and angle.
        
        \item When changing zoom level, all the markers shall be deleted and re-drawn on the screen, in order to correctly show obstacles' distance in the current zoom level scale. This shall however still be compliant with what expressed in point \ref{marker-shown-obst} of user requirements.
    
    \end{enumerate}
    
\end{enumerate}

\section{Working Modes}

This section illustrates all working modes.

\subsection{Calibration Mode}

\begin{enumerate}
    \item \textbf{Entering mode}\\
        System enters this mode as soon as it is turned on or at reset.
    \item \textbf{While in mode}
        \begin{enumerate}
            \item
            \wmodeact{}{User presses B\_ZOOM.}{}{System is now in Scanning Mode.}
            \item
            \wmodeact{}{User presses B\_RESET.}{}{System shuts down and resets itself.}
        \end{enumerate}
        
\end{enumerate}
    

\subsection{Scanning Mode}

\begin{enumerate}
    \item \textbf{Entering mode}\\
        System enters this mode as soon as user presses B\_ZOOM during Calibration Mode.
    
    \item \textbf{While in mode}
        \begin{enumerate}
            \item
            \wmodeact{Current zoom level is different from the maximum one -- which is 5.}{user presses the B\_ZOOM.}{}{
            Current zoom level is equal to the one before this action + 1.}
            
            \item
            \wmodeact{Current zoom level is equal to the maximum one -- which is 5.}{User presses B\_ZOOM.}{}{Current zoom level is equal to the initial one -- which is 0.}
            
            \item
            \wmodeact{}{User presses B\_RESET.}{}{System shuts down and resets itself.}
        \end{enumerate}
\end{enumerate}